% Options for packages loaded elsewhere
\PassOptionsToPackage{unicode}{hyperref}
\PassOptionsToPackage{hyphens}{url}
%
\documentclass[
  11pt,
]{article}
\usepackage{amsmath,amssymb}
\usepackage{iftex}
\ifPDFTeX
  \usepackage[T1]{fontenc}
  \usepackage[utf8]{inputenc}
  \usepackage{textcomp} % provide euro and other symbols
\else % if luatex or xetex
  \usepackage{unicode-math} % this also loads fontspec
  \defaultfontfeatures{Scale=MatchLowercase}
  \defaultfontfeatures[\rmfamily]{Ligatures=TeX,Scale=1}
\fi
\usepackage{lmodern}
\ifPDFTeX\else
  % xetex/luatex font selection
\fi
% Use upquote if available, for straight quotes in verbatim environments
\IfFileExists{upquote.sty}{\usepackage{upquote}}{}
\IfFileExists{microtype.sty}{% use microtype if available
  \usepackage[]{microtype}
  \UseMicrotypeSet[protrusion]{basicmath} % disable protrusion for tt fonts
}{}
\makeatletter
\@ifundefined{KOMAClassName}{% if non-KOMA class
  \IfFileExists{parskip.sty}{%
    \usepackage{parskip}
  }{% else
    \setlength{\parindent}{0pt}
    \setlength{\parskip}{6pt plus 2pt minus 1pt}}
}{% if KOMA class
  \KOMAoptions{parskip=half}}
\makeatother
\usepackage{xcolor}
\usepackage[margin=1in]{geometry}
\usepackage{color}
\usepackage{fancyvrb}
\newcommand{\VerbBar}{|}
\newcommand{\VERB}{\Verb[commandchars=\\\{\}]}
\DefineVerbatimEnvironment{Highlighting}{Verbatim}{commandchars=\\\{\}}
% Add ',fontsize=\small' for more characters per line
\usepackage{framed}
\definecolor{shadecolor}{RGB}{248,248,248}
\newenvironment{Shaded}{\begin{snugshade}}{\end{snugshade}}
\newcommand{\AlertTok}[1]{\textcolor[rgb]{0.94,0.16,0.16}{#1}}
\newcommand{\AnnotationTok}[1]{\textcolor[rgb]{0.56,0.35,0.01}{\textbf{\textit{#1}}}}
\newcommand{\AttributeTok}[1]{\textcolor[rgb]{0.13,0.29,0.53}{#1}}
\newcommand{\BaseNTok}[1]{\textcolor[rgb]{0.00,0.00,0.81}{#1}}
\newcommand{\BuiltInTok}[1]{#1}
\newcommand{\CharTok}[1]{\textcolor[rgb]{0.31,0.60,0.02}{#1}}
\newcommand{\CommentTok}[1]{\textcolor[rgb]{0.56,0.35,0.01}{\textit{#1}}}
\newcommand{\CommentVarTok}[1]{\textcolor[rgb]{0.56,0.35,0.01}{\textbf{\textit{#1}}}}
\newcommand{\ConstantTok}[1]{\textcolor[rgb]{0.56,0.35,0.01}{#1}}
\newcommand{\ControlFlowTok}[1]{\textcolor[rgb]{0.13,0.29,0.53}{\textbf{#1}}}
\newcommand{\DataTypeTok}[1]{\textcolor[rgb]{0.13,0.29,0.53}{#1}}
\newcommand{\DecValTok}[1]{\textcolor[rgb]{0.00,0.00,0.81}{#1}}
\newcommand{\DocumentationTok}[1]{\textcolor[rgb]{0.56,0.35,0.01}{\textbf{\textit{#1}}}}
\newcommand{\ErrorTok}[1]{\textcolor[rgb]{0.64,0.00,0.00}{\textbf{#1}}}
\newcommand{\ExtensionTok}[1]{#1}
\newcommand{\FloatTok}[1]{\textcolor[rgb]{0.00,0.00,0.81}{#1}}
\newcommand{\FunctionTok}[1]{\textcolor[rgb]{0.13,0.29,0.53}{\textbf{#1}}}
\newcommand{\ImportTok}[1]{#1}
\newcommand{\InformationTok}[1]{\textcolor[rgb]{0.56,0.35,0.01}{\textbf{\textit{#1}}}}
\newcommand{\KeywordTok}[1]{\textcolor[rgb]{0.13,0.29,0.53}{\textbf{#1}}}
\newcommand{\NormalTok}[1]{#1}
\newcommand{\OperatorTok}[1]{\textcolor[rgb]{0.81,0.36,0.00}{\textbf{#1}}}
\newcommand{\OtherTok}[1]{\textcolor[rgb]{0.56,0.35,0.01}{#1}}
\newcommand{\PreprocessorTok}[1]{\textcolor[rgb]{0.56,0.35,0.01}{\textit{#1}}}
\newcommand{\RegionMarkerTok}[1]{#1}
\newcommand{\SpecialCharTok}[1]{\textcolor[rgb]{0.81,0.36,0.00}{\textbf{#1}}}
\newcommand{\SpecialStringTok}[1]{\textcolor[rgb]{0.31,0.60,0.02}{#1}}
\newcommand{\StringTok}[1]{\textcolor[rgb]{0.31,0.60,0.02}{#1}}
\newcommand{\VariableTok}[1]{\textcolor[rgb]{0.00,0.00,0.00}{#1}}
\newcommand{\VerbatimStringTok}[1]{\textcolor[rgb]{0.31,0.60,0.02}{#1}}
\newcommand{\WarningTok}[1]{\textcolor[rgb]{0.56,0.35,0.01}{\textbf{\textit{#1}}}}
\usepackage{graphicx}
\makeatletter
\def\maxwidth{\ifdim\Gin@nat@width>\linewidth\linewidth\else\Gin@nat@width\fi}
\def\maxheight{\ifdim\Gin@nat@height>\textheight\textheight\else\Gin@nat@height\fi}
\makeatother
% Scale images if necessary, so that they will not overflow the page
% margins by default, and it is still possible to overwrite the defaults
% using explicit options in \includegraphics[width, height, ...]{}
\setkeys{Gin}{width=\maxwidth,height=\maxheight,keepaspectratio}
% Set default figure placement to htbp
\makeatletter
\def\fps@figure{htbp}
\makeatother
\setlength{\emergencystretch}{3em} % prevent overfull lines
\providecommand{\tightlist}{%
  \setlength{\itemsep}{0pt}\setlength{\parskip}{0pt}}
\setcounter{secnumdepth}{5}
\ifLuaTeX
  \usepackage{selnolig}  % disable illegal ligatures
\fi
\IfFileExists{bookmark.sty}{\usepackage{bookmark}}{\usepackage{hyperref}}
\IfFileExists{xurl.sty}{\usepackage{xurl}}{} % add URL line breaks if available
\urlstyle{same}
\hypersetup{
  pdftitle={PBD Assignment 01},
  pdfauthor={Md Mobashir Rahman and Adwitiya Argha Priyadarshini Boruah},
  hidelinks,
  pdfcreator={LaTeX via pandoc}}

\title{PBD Assignment 01}
\author{Md Mobashir Rahman and Adwitiya Argha Priyadarshini Boruah}
\date{November 04, 2025}

\begin{document}
\maketitle

\fbox{
\begin{minipage}{0.9\textwidth}
\textbf{Author:} Md Mobashir Rahman and Adwitiya Argha Priyadarshini Boruah \\
\textbf{Email:} mdra00001@stud.uni-saarland.de, adbo00002@stud.uni-saarland.de \\
\textbf{Matriculation Number:} 7059086, 7070291
\end{minipage}
}

\hypertarget{principal-component-analysis-pca-and-data-imputation}{%
\section{Principal Component Analysis (PCA) and Data
Imputation}\label{principal-component-analysis-pca-and-data-imputation}}

\hypertarget{exercise-principal-component-analysis-40-points}{%
\subsection{Exercise: Principal Component Analysis (40
points)}\label{exercise-principal-component-analysis-40-points}}

Write a program that applies the PCA technique to the toy dataset in the
supplement (``pca toy.txt'').

\textbf{-\textgreater{} (a): }\textbf{Standardize the variables in the
dataset. Why is this step necessary before performing a PCA?}

\hypertarget{a-standardize-the-variables-before-running-pca}{%
\subsubsection{(a) Standardize the variables before running
PCA:}\label{a-standardize-the-variables-before-running-pca}}

We first load the toy dataset and standardize each variable to have mean
0 and standard deviation 1. Standardization is necessary so that
variables measured on different scales contribute equally to the
principal components; otherwise, variables with larger variances would
dominate the PCA solution.

\begin{Shaded}
\begin{Highlighting}[]
\NormalTok{pca\_raw }\OtherTok{\textless{}{-}} \FunctionTok{read.delim}\NormalTok{(}\StringTok{"Assignment\_1\_supplement/pca\_toy.txt"}\NormalTok{, }\AttributeTok{check.names =} \ConstantTok{FALSE}\NormalTok{)}
\NormalTok{pca\_scaled }\OtherTok{\textless{}{-}} \FunctionTok{scale}\NormalTok{(pca\_raw)}
\NormalTok{pca\_summary }\OtherTok{\textless{}{-}} \FunctionTok{data.frame}\NormalTok{(}
  \AttributeTok{variable =} \FunctionTok{colnames}\NormalTok{(pca\_raw),}
  \AttributeTok{mean =} \FunctionTok{apply}\NormalTok{(pca\_scaled, }\DecValTok{2}\NormalTok{, mean),}
  \AttributeTok{sd =} \FunctionTok{apply}\NormalTok{(pca\_scaled, }\DecValTok{2}\NormalTok{, sd)}
\NormalTok{)}
\NormalTok{pca\_summary}
\end{Highlighting}
\end{Shaded}

\begin{verbatim}
##   variable          mean sd
## a        a -4.950207e-16  1
## b        b  1.920686e-16  1
## c        c -1.143530e-16  1
## d        d  3.486100e-15  1
\end{verbatim}

\textbf{Observation}: Column means are \textasciitilde0 and SDs are 1
(tiny numerical noise is expected).

\textbf{-\textgreater{} (b): }\textbf{Create a scatter plot from the
transformed data, with PC1 and PC2 as the axes.}

\hypertarget{b-run-pca-and-create-a-scatter-plot-of-pc1-vs-pc2}{%
\subsubsection{(b) Run PCA and create a scatter plot of PC1 vs
PC2:}\label{b-run-pca-and-create-a-scatter-plot-of-pc1-vs-pc2}}

We compute the PCA on standardized variables and plot the scores on PC1
vs PC2.

\begin{Shaded}
\begin{Highlighting}[]
\NormalTok{pca\_model }\OtherTok{\textless{}{-}} \FunctionTok{prcomp}\NormalTok{(pca\_scaled, }\AttributeTok{center =} \ConstantTok{FALSE}\NormalTok{, }\AttributeTok{scale. =} \ConstantTok{FALSE}\NormalTok{)}
\NormalTok{pca\_scores }\OtherTok{\textless{}{-}} \FunctionTok{as.data.frame}\NormalTok{(pca\_model}\SpecialCharTok{$}\NormalTok{x[, }\DecValTok{1}\SpecialCharTok{:}\DecValTok{2}\NormalTok{])}
\end{Highlighting}
\end{Shaded}

\begin{Shaded}
\begin{Highlighting}[]
\FunctionTok{plot}\NormalTok{(}
\NormalTok{  pca\_scores}\SpecialCharTok{$}\NormalTok{PC1,}
\NormalTok{  pca\_scores}\SpecialCharTok{$}\NormalTok{PC2,}
  \AttributeTok{xlab =} \StringTok{"PC1"}\NormalTok{,}
  \AttributeTok{ylab =} \StringTok{"PC2"}\NormalTok{,}
  \AttributeTok{main =} \StringTok{"PCA of Toy Dataset: PC1 vs PC2 Scores (Standardized Data)"}\NormalTok{,}
  \AttributeTok{pch =} \DecValTok{19}\NormalTok{,}
  \AttributeTok{col =} \StringTok{"steelblue"}
\NormalTok{)}
\FunctionTok{grid}\NormalTok{()}
\end{Highlighting}
\end{Shaded}

\begin{center}\includegraphics{assignment01_files/figure-latex/pca_scatter_plot-1} \end{center}

\textbf{-\textgreater{} (c): }\textbf{Which variables are the most
important for PC1 and PC2? Why?}

\hypertarget{c-identify-the-most-important-variables-for-pc1-and-pc2}{%
\subsubsection{(c) Identify the most important variables for PC1 and
PC2:}\label{c-identify-the-most-important-variables-for-pc1-and-pc2}}

We examine the loadings (i.e., the coefficients of the eigenvectors) for
PC1 and PC2. Variables with larger absolute loadings contribute more to
the corresponding component.

\begin{Shaded}
\begin{Highlighting}[]
\NormalTok{pca\_loadings }\OtherTok{\textless{}{-}}\NormalTok{ pca\_model}\SpecialCharTok{$}\NormalTok{rotation[, }\DecValTok{1}\SpecialCharTok{:}\DecValTok{2}\NormalTok{]}
\NormalTok{pca\_loadings}
\end{Highlighting}
\end{Shaded}

\begin{verbatim}
##          PC1        PC2
## a -0.4357230  0.5053838
## b -0.6113755 -0.3355008
## c  0.2745604 -0.7132847
## d -0.6008179 -0.3510717
\end{verbatim}

From the loadings we can identify the variables with the largest
absolute values for each component. PC1 is most influenced by b and d,
while PC2 is most influenced by c and a. These variables have the
highest leverage in shaping the corresponding components.

\textbf{-\textgreater{} (d) :}\textbf{What percentage of the variance in
the dataset is explained by PC1 and PC2?}

\hypertarget{d-determine-the-variance-explained-by-pc1-and-pc2}{%
\subsubsection{(d) Determine the variance explained by PC1 and
PC2:}\label{d-determine-the-variance-explained-by-pc1-and-pc2}}

We compute the proportion of variance explained by each principal
component and report the first two.

\begin{Shaded}
\begin{Highlighting}[]
\NormalTok{importance\_table }\OtherTok{\textless{}{-}} \FunctionTok{summary}\NormalTok{(pca\_model)}\SpecialCharTok{$}\NormalTok{importance}
\NormalTok{importance\_table}
\end{Highlighting}
\end{Shaded}

\begin{verbatim}
##                            PC1      PC2       PC3       PC4
## Standard deviation     1.47734 1.116475 0.6980192 0.2893467
## Proportion of Variance 0.54563 0.311630 0.1218100 0.0209300
## Cumulative Proportion  0.54563 0.857260 0.9790700 1.0000000
\end{verbatim}

PC1 and PC2 together explain 85.73\% of the variance in the standardized
dataset.

\textbf{-\textgreater{} (e): }\textbf{Under what circumstances would
more PCs be considered?}

\hypertarget{e-when-to-consider-more-principal-components}{%
\subsubsection{(e) When to consider more principal
components:}\label{e-when-to-consider-more-principal-components}}

We would consider additional principal components if the first two
components do not capture a sufficient proportion of the total variance
for the analysis goals, or if later components reveal important
structure or patterns (e.g., clustering) relevant to the research
question. Scree plots, cumulative variance thresholds (such as
80\%--90\%), or domain knowledge about necessary detail can guide this
decision.

\hypertarget{exercise-data-imputation-60-points}{%
\subsection{Exercise: Data Imputation (60
points)}\label{exercise-data-imputation-60-points}}

Data imputation techniques can be used to fill out missing data in
sparse dataframes. Here, we will try to generate missing entries in a
proteomics dataset (''ms toy.txt'') that were below the detection limit
for expression data. Missing values are denoted as ''NA''.

\hypertarget{imputation-based-on-a-given-data-distribution-30-points}{%
\subsubsection{1) Imputation based on a given data distribution (30
points)}\label{imputation-based-on-a-given-data-distribution-30-points}}

\textbf{-\textgreater{} (a): }\textbf{Write a function that imputes
missing values by sampling from a normal distribution whose mean comes
from a lower quantile of the observed data and whose standard deviation
is a fraction of the observed SD.}

\hypertarget{a-implement-the-imputation-function}{%
\paragraph{(a) Implement the imputation
function:}\label{a-implement-the-imputation-function}}

The function below takes a numeric vector with NAs, computes the
observed mean and SD, sets the imputation mean to
\texttt{qnorm(percentile,\ mean,\ sd)} (lower tail), and the imputation
SD to \texttt{sd\_ratio\ *\ sd}. It draws imputed values with
\texttt{rnorm} and fills the missing entries.

\begin{Shaded}
\begin{Highlighting}[]
\CommentTok{\# Imputation based on a given data distribution}
\NormalTok{impute\_from\_distribution }\OtherTok{\textless{}{-}} \ControlFlowTok{function}\NormalTok{(x, }\AttributeTok{percentile =} \FloatTok{0.05}\NormalTok{, }\AttributeTok{sd\_ratio =} \FloatTok{0.5}\NormalTok{, }\AttributeTok{seed =} \DecValTok{42}\NormalTok{) \{}
  
  \CommentTok{\# {-}{-}{-} Step 0: Setup and preprocessing {-}{-}{-}}
  
  \CommentTok{\# set a random seed for reproducibility}
  \ControlFlowTok{if}\NormalTok{ (}\SpecialCharTok{!}\FunctionTok{is.null}\NormalTok{(seed)) }\FunctionTok{set.seed}\NormalTok{(seed)}
  
  \CommentTok{\# Separate observed (non{-}missing) values}
\NormalTok{  obs }\OtherTok{\textless{}{-}}\NormalTok{ x[}\SpecialCharTok{!}\FunctionTok{is.na}\NormalTok{(x)]}
  
  \CommentTok{\# {-}{-}{-} Step 1: Calculate mean and standard deviation of observed data {-}{-}{-}}
\NormalTok{  m }\OtherTok{\textless{}{-}} \FunctionTok{mean}\NormalTok{(obs)}
\NormalTok{  s }\OtherTok{\textless{}{-}} \FunctionTok{sd}\NormalTok{(obs)}
  
  \CommentTok{\# {-}{-}{-} Step 2: Derive new mean from lower quantile of the old distribution {-}{-}{-}}
  \CommentTok{\# We use qnorm to get the value at a given lower quantile (e.g., 5\%)}
  \CommentTok{\# This simulates the lower tail of the original distribution for imputation}
\NormalTok{  mu\_imp }\OtherTok{\textless{}{-}} \FunctionTok{qnorm}\NormalTok{(percentile, }\AttributeTok{mean =}\NormalTok{ m, }\AttributeTok{sd =}\NormalTok{ s)}
  
  \CommentTok{\# {-}{-}{-} Step 3: Derive new standard deviation as a fraction of observed SD {-}{-}{-}}
  \CommentTok{\# This ensures the imputed values vary less than the observed data}
\NormalTok{  sd\_imp }\OtherTok{\textless{}{-}}\NormalTok{ sd\_ratio }\SpecialCharTok{*}\NormalTok{ s}
  
  \CommentTok{\# {-}{-}{-} Step 4: Generate imputed values for missing entries {-}{-}{-}}
\NormalTok{  n\_miss }\OtherTok{\textless{}{-}} \FunctionTok{sum}\NormalTok{(}\FunctionTok{is.na}\NormalTok{(x))}
\NormalTok{  imp\_vals }\OtherTok{\textless{}{-}} \FunctionTok{rnorm}\NormalTok{(n\_miss, }\AttributeTok{mean =}\NormalTok{ mu\_imp, }\AttributeTok{sd =}\NormalTok{ sd\_imp)}
  
  \CommentTok{\# Replace missing values with imputed values}
\NormalTok{  x\_imp }\OtherTok{\textless{}{-}}\NormalTok{ x}
\NormalTok{  x\_imp[}\FunctionTok{is.na}\NormalTok{(x\_imp)] }\OtherTok{\textless{}{-}}\NormalTok{ imp\_vals}
  
  \CommentTok{\# {-}{-}{-} Return results and summary statistics {-}{-}{-}}
  \FunctionTok{return}\NormalTok{(}\FunctionTok{list}\NormalTok{(}
    \AttributeTok{values =}\NormalTok{ x\_imp,      }\CommentTok{\# full vector with imputed values}
    \AttributeTok{imputed =}\NormalTok{ imp\_vals,  }\CommentTok{\# the imputed subset}
    \AttributeTok{mu\_imp =}\NormalTok{ mu\_imp,     }\CommentTok{\# imputation mean}
    \AttributeTok{sd\_imp =}\NormalTok{ sd\_imp,     }\CommentTok{\# imputation standard deviation}
    \AttributeTok{mean\_obs =}\NormalTok{ m,        }\CommentTok{\# observed mean}
    \AttributeTok{sd\_obs =}\NormalTok{ s,          }\CommentTok{\# observed standard deviation}
    \AttributeTok{n\_missing =}\NormalTok{ n\_miss   }\CommentTok{\# number of missing values imputed}
\NormalTok{  ))}
\NormalTok{\}}
\end{Highlighting}
\end{Shaded}

\textbf{-\textgreater{} (b): }\textbf{Apply your function to the
variable ctrl.1 and plot the overall sample and the imputed data
similarly to Figure 1.}

\hypertarget{b-apply-to-ctrl.1-and-produce-a-plot}{%
\paragraph{(b) Apply to ctrl.1 and produce a
plot:}\label{b-apply-to-ctrl.1-and-produce-a-plot}}

We read the dataset, impute \texttt{ctrl.1} using a lower-tail mean
(\texttt{percentile\ =\ 0.05}) and a reduced spread
(\texttt{sd\_ratio\ =\ 0.5}), and overlay the histograms of the overall
values (blue) and imputed values (red).

\begin{Shaded}
\begin{Highlighting}[]
\NormalTok{ms\_raw }\OtherTok{\textless{}{-}} \FunctionTok{read.delim}\NormalTok{(}\StringTok{"Assignment\_1\_supplement/ms\_toy.txt"}\NormalTok{, }\AttributeTok{check.names =} \ConstantTok{FALSE}\NormalTok{)}
\NormalTok{ctrl1 }\OtherTok{\textless{}{-}}\NormalTok{ ms\_raw}\SpecialCharTok{$}\NormalTok{ctrl}\FloatTok{.1}
\FunctionTok{sum}\NormalTok{(}\FunctionTok{is.na}\NormalTok{(ctrl1))  }\CommentTok{\# missing count before}
\end{Highlighting}
\end{Shaded}

\begin{verbatim}
## [1] 587
\end{verbatim}

\begin{Shaded}
\begin{Highlighting}[]
\NormalTok{imp\_ctrl1 }\OtherTok{\textless{}{-}} \FunctionTok{impute\_from\_distribution}\NormalTok{(ctrl1, }\AttributeTok{percentile =} \FloatTok{0.05}\NormalTok{, }\AttributeTok{sd\_ratio =} \FloatTok{0.5}\NormalTok{, }\AttributeTok{seed =} \DecValTok{123}\NormalTok{)}
\FunctionTok{sum}\NormalTok{(}\FunctionTok{is.na}\NormalTok{(imp\_ctrl1}\SpecialCharTok{$}\NormalTok{values))  }\CommentTok{\# missing count after}
\end{Highlighting}
\end{Shaded}

\begin{verbatim}
## [1] 0
\end{verbatim}

\begin{Shaded}
\begin{Highlighting}[]
\NormalTok{all\_vals }\OtherTok{\textless{}{-}}\NormalTok{ imp\_ctrl1}\SpecialCharTok{$}\NormalTok{values}
\NormalTok{imp\_vals }\OtherTok{\textless{}{-}}\NormalTok{ imp\_ctrl1}\SpecialCharTok{$}\NormalTok{imputed}
\NormalTok{brks }\OtherTok{\textless{}{-}} \FunctionTok{seq}\NormalTok{(}\FunctionTok{floor}\NormalTok{(}\FunctionTok{min}\NormalTok{(all\_vals)), }\FunctionTok{ceiling}\NormalTok{(}\FunctionTok{max}\NormalTok{(all\_vals)), }\AttributeTok{length.out =} \DecValTok{60}\NormalTok{)}

\FunctionTok{hist}\NormalTok{(all\_vals,}
     \AttributeTok{breaks =}\NormalTok{ brks,}
     \AttributeTok{col =} \FunctionTok{rgb}\NormalTok{(}\DecValTok{0}\NormalTok{, }\DecValTok{0}\NormalTok{, }\DecValTok{1}\NormalTok{, }\FloatTok{0.7}\NormalTok{), }\AttributeTok{border =} \ConstantTok{NA}\NormalTok{,}
     \AttributeTok{main =} \StringTok{"Imputation for ctrl.1: Overall (Blue) vs Imputed (Red)"}\NormalTok{,}
     \AttributeTok{xlab =} \StringTok{"Intensity"}\NormalTok{)}
\FunctionTok{hist}\NormalTok{(imp\_vals,}
     \AttributeTok{breaks =}\NormalTok{ brks,}
     \AttributeTok{col =} \FunctionTok{rgb}\NormalTok{(}\DecValTok{1}\NormalTok{, }\DecValTok{0}\NormalTok{, }\DecValTok{0}\NormalTok{, }\FloatTok{0.7}\NormalTok{), }\AttributeTok{border =} \ConstantTok{NA}\NormalTok{,}
     \AttributeTok{add =} \ConstantTok{TRUE}\NormalTok{)}
\FunctionTok{legend}\NormalTok{(}\StringTok{"topright"}\NormalTok{, }\AttributeTok{inset =} \FloatTok{0.02}\NormalTok{,}
       \AttributeTok{legend =} \FunctionTok{c}\NormalTok{(}\StringTok{"Overall (after imputation)"}\NormalTok{, }\StringTok{"Imputed only"}\NormalTok{),}
       \AttributeTok{fill =} \FunctionTok{c}\NormalTok{(}\FunctionTok{rgb}\NormalTok{(}\DecValTok{0}\NormalTok{, }\DecValTok{0}\NormalTok{, }\DecValTok{1}\NormalTok{, }\FloatTok{0.7}\NormalTok{), }\FunctionTok{rgb}\NormalTok{(}\DecValTok{1}\NormalTok{, }\DecValTok{0}\NormalTok{, }\DecValTok{0}\NormalTok{, }\FloatTok{0.7}\NormalTok{)), }\AttributeTok{border =} \ConstantTok{NA}\NormalTok{, }\AttributeTok{cex =} \FloatTok{0.8}\NormalTok{)}
\FunctionTok{grid}\NormalTok{()}
\end{Highlighting}
\end{Shaded}

\begin{center}\includegraphics{assignment01_files/figure-latex/impute_plot_ctrl1-1} \end{center}

\begin{Shaded}
\begin{Highlighting}[]
\FunctionTok{data.frame}\NormalTok{(}
  \AttributeTok{observed\_mean =} \FunctionTok{round}\NormalTok{(imp\_ctrl1}\SpecialCharTok{$}\NormalTok{mean\_obs, }\DecValTok{3}\NormalTok{),}
  \AttributeTok{observed\_sd =} \FunctionTok{round}\NormalTok{(imp\_ctrl1}\SpecialCharTok{$}\NormalTok{sd\_obs, }\DecValTok{3}\NormalTok{),}
  \AttributeTok{imputed\_mean =} \FunctionTok{round}\NormalTok{(imp\_ctrl1}\SpecialCharTok{$}\NormalTok{mu\_imp, }\DecValTok{3}\NormalTok{),}
  \AttributeTok{imputed\_sd =} \FunctionTok{round}\NormalTok{(imp\_ctrl1}\SpecialCharTok{$}\NormalTok{sd\_imp, }\DecValTok{3}\NormalTok{),}
  \AttributeTok{n\_missing =}\NormalTok{ imp\_ctrl1}\SpecialCharTok{$}\NormalTok{n\_missing}
\NormalTok{)}
\end{Highlighting}
\end{Shaded}

\begin{verbatim}
##   observed_mean observed_sd imputed_mean imputed_sd n_missing
## 1        29.036       2.739       24.531       1.37       587
\end{verbatim}

\textbf{-\textgreater{} (c): }\textbf{What is the effect of changing the
percentile parameter in Step 2 and the ratio parameter in Step 3?}

\hypertarget{c-parameter-effects}{%
\paragraph{(c) Parameter effects:}\label{c-parameter-effects}}

Lower percentiles move the imputed mean further into the left tail
(smaller values), while larger percentiles shift imputed values upward.
Smaller SD ratios produce a tighter imputed distribution (less spread),
while larger ratios widen it. The table shows the imputation mean and SD
for several settings.

\begin{Shaded}
\begin{Highlighting}[]
\NormalTok{percentiles }\OtherTok{\textless{}{-}} \FunctionTok{c}\NormalTok{(}\FloatTok{0.01}\NormalTok{, }\FloatTok{0.05}\NormalTok{, }\FloatTok{0.10}\NormalTok{)}
\NormalTok{ratios }\OtherTok{\textless{}{-}} \FunctionTok{c}\NormalTok{(}\FloatTok{0.2}\NormalTok{, }\FloatTok{0.4}\NormalTok{, }\FloatTok{0.6}\NormalTok{)}
\NormalTok{out }\OtherTok{\textless{}{-}} \FunctionTok{list}\NormalTok{()}
\ControlFlowTok{for}\NormalTok{ (p }\ControlFlowTok{in}\NormalTok{ percentiles) \{}
  \ControlFlowTok{for}\NormalTok{ (r }\ControlFlowTok{in}\NormalTok{ ratios) \{}
\NormalTok{    tmp }\OtherTok{\textless{}{-}} \FunctionTok{impute\_from\_distribution}\NormalTok{(ctrl1, }\AttributeTok{percentile =}\NormalTok{ p, }\AttributeTok{sd\_ratio =}\NormalTok{ r, }\AttributeTok{seed =} \DecValTok{123}\NormalTok{)}
\NormalTok{    out[[}\FunctionTok{length}\NormalTok{(out) }\SpecialCharTok{+} \DecValTok{1}\NormalTok{]] }\OtherTok{\textless{}{-}} \FunctionTok{data.frame}\NormalTok{(}
      \AttributeTok{percentile =}\NormalTok{ p,}
      \AttributeTok{sd\_ratio =}\NormalTok{ r,}
      \AttributeTok{imputed\_mean =}\NormalTok{ tmp}\SpecialCharTok{$}\NormalTok{mu\_imp,}
      \AttributeTok{imputed\_sd =}\NormalTok{ tmp}\SpecialCharTok{$}\NormalTok{sd\_imp}
\NormalTok{    )}
\NormalTok{  \}}
\NormalTok{\}}
\FunctionTok{do.call}\NormalTok{(rbind, out)}
\end{Highlighting}
\end{Shaded}

\begin{verbatim}
##   percentile sd_ratio imputed_mean imputed_sd
## 1       0.01      0.2     22.66407  0.5478088
## 2       0.01      0.4     22.66407  1.0956176
## 3       0.01      0.6     22.66407  1.6434264
## 4       0.05      0.2     24.53071  0.5478088
## 5       0.05      0.4     24.53071  1.0956176
## 6       0.05      0.6     24.53071  1.6434264
## 7       0.10      0.2     25.52581  0.5478088
## 8       0.10      0.4     25.52581  1.0956176
## 9       0.10      0.6     25.52581  1.6434264
\end{verbatim}

Visualization: Histograms showing how the imputed distribution changes
with different \texttt{percentile} and \texttt{sd\_ratio} settings.

\begin{center}\includegraphics{assignment01_files/figure-latex/impute_param_hist_grid_patchwork-1} \end{center}

\textbf{-\textgreater{} (d): }\textbf{Which configuration of parameters
is the most logical/desirable?}

\hypertarget{d-recommended-configuration}{%
\paragraph{(d) Recommended
configuration:}\label{d-recommended-configuration}}

The goal is to impute data that was ``below the detection limit,'' which
implies the true values are both low and consistent. The parameter p
controls the center of the imputed (red) data, and a small p like 0.01
correctly shifts this mean to the low-intensity region, separate from
the observed (blue) data. The parameter r controls the spread, and a
small r like 0.2 creates a tight, narrow peak. (see the plots above)
This narrow peak logically represents a group of values that all failed
to cross the same detection threshold. Therefore, combinations like
(p=0.01, r=0.2) is one of the combination that creates the distinct,
low, and non-variable cluster that we would expect from such missing
data.

\hypertarget{k-nearest-neighbor-imputation-30-points}{%
\subsubsection{2) k-Nearest Neighbor imputation (30
points)}\label{k-nearest-neighbor-imputation-30-points}}

\textbf{-\textgreater{} (a): }\textbf{Create a dataset with only ctrl.1,
ctrl.2, ctrl.3; remove rows where all three are missing; then perform
kNN imputation.}

\hypertarget{a-prepare-control-only-data-and-run-knn}{%
\paragraph{(a) Prepare control-only data and run
kNN:}\label{a-prepare-control-only-data-and-run-knn}}

We subset the three control samples and drop rows with no information
across all three controls. We then use \texttt{VIM::kNN} with
\texttt{k\ =\ 5}.

\begin{Shaded}
\begin{Highlighting}[]
\NormalTok{ms\_ctrl }\OtherTok{\textless{}{-}}\NormalTok{ ms\_raw[, }\FunctionTok{c}\NormalTok{(}\StringTok{"ctrl.1"}\NormalTok{, }\StringTok{"ctrl.2"}\NormalTok{, }\StringTok{"ctrl.3"}\NormalTok{)]}
\NormalTok{all\_three\_na }\OtherTok{\textless{}{-}} \FunctionTok{rowSums}\NormalTok{(}\FunctionTok{is.na}\NormalTok{(ms\_ctrl)) }\SpecialCharTok{==} \FunctionTok{ncol}\NormalTok{(ms\_ctrl)}
\NormalTok{ms\_ctrl\_filt }\OtherTok{\textless{}{-}}\NormalTok{ ms\_ctrl[}\SpecialCharTok{!}\NormalTok{all\_three\_na, ]}

\NormalTok{missing\_before\_ctrl1 }\OtherTok{\textless{}{-}} \FunctionTok{sum}\NormalTok{(}\FunctionTok{is.na}\NormalTok{(ms\_ctrl\_filt}\SpecialCharTok{$}\NormalTok{ctrl}\FloatTok{.1}\NormalTok{))}
\FunctionTok{list}\NormalTok{(}\AttributeTok{rows\_after\_filter =} \FunctionTok{nrow}\NormalTok{(ms\_ctrl\_filt), }\AttributeTok{missing\_ctrl1\_before =}\NormalTok{ missing\_before\_ctrl1)}
\end{Highlighting}
\end{Shaded}

\begin{verbatim}
## $rows_after_filter
## [1] 6733
## 
## $missing_ctrl1_before
## [1] 360
\end{verbatim}

\begin{Shaded}
\begin{Highlighting}[]
\CommentTok{\# load VIM}
\FunctionTok{suppressPackageStartupMessages}\NormalTok{(}\FunctionTok{library}\NormalTok{(VIM))}
\NormalTok{knn\_res }\OtherTok{\textless{}{-}}\NormalTok{ VIM}\SpecialCharTok{::}\FunctionTok{kNN}\NormalTok{(ms\_ctrl\_filt, }\AttributeTok{k =} \DecValTok{5}\NormalTok{, }\AttributeTok{imp\_var =} \ConstantTok{TRUE}\NormalTok{)}

\CommentTok{\# Extract imputed indicator and values for ctrl.1}
\NormalTok{imp\_flag\_ctrl1 }\OtherTok{\textless{}{-}}\NormalTok{ knn\_res}\SpecialCharTok{$}\NormalTok{ctrl}\FloatTok{.1}\NormalTok{\_imp}
\ControlFlowTok{if}\NormalTok{ (}\SpecialCharTok{!}\FunctionTok{is.logical}\NormalTok{(imp\_flag\_ctrl1)) imp\_flag\_ctrl1 }\OtherTok{\textless{}{-}} \FunctionTok{as.logical}\NormalTok{(imp\_flag\_ctrl1)}

\NormalTok{ctrl1\_knn\_all }\OtherTok{\textless{}{-}}\NormalTok{ knn\_res}\SpecialCharTok{$}\NormalTok{ctrl}\FloatTok{.1}
\NormalTok{ctrl1\_knn\_imp }\OtherTok{\textless{}{-}}\NormalTok{ ctrl1\_knn\_all[imp\_flag\_ctrl1]}

\FunctionTok{list}\NormalTok{(}
  \AttributeTok{missing\_ctrl1\_after =} \FunctionTok{sum}\NormalTok{(}\FunctionTok{is.na}\NormalTok{(ctrl1\_knn\_all)),}
  \AttributeTok{imputed\_count\_ctrl1 =} \FunctionTok{length}\NormalTok{(ctrl1\_knn\_imp)}
\NormalTok{)}
\end{Highlighting}
\end{Shaded}

\begin{verbatim}
## $missing_ctrl1_after
## [1] 0
## 
## $imputed_count_ctrl1
## [1] 360
\end{verbatim}

\textbf{-\textgreater{} (b): }\textbf{Plot the imputed data for ctrl.1
similarly to Figure 1 and compare its distribution.}

\hypertarget{b-plot-ctrl.1-after-knn-imputation}{%
\paragraph{(b) Plot ctrl.1 after kNN
imputation:}\label{b-plot-ctrl.1-after-knn-imputation}}

We overlay the histogram of all ctrl.1 values after kNN imputation
(blue) with the histogram of imputed-only values (red).

\begin{Shaded}
\begin{Highlighting}[]
\NormalTok{brks2 }\OtherTok{\textless{}{-}} \FunctionTok{seq}\NormalTok{(}\FunctionTok{floor}\NormalTok{(}\FunctionTok{min}\NormalTok{(ctrl1\_knn\_all)), }\FunctionTok{ceiling}\NormalTok{(}\FunctionTok{max}\NormalTok{(ctrl1\_knn\_all)), }\AttributeTok{length.out =} \DecValTok{60}\NormalTok{)}
\FunctionTok{hist}\NormalTok{(ctrl1\_knn\_all,}
     \AttributeTok{breaks =}\NormalTok{ brks2,}
     \AttributeTok{col =} \FunctionTok{rgb}\NormalTok{(}\DecValTok{0}\NormalTok{, }\DecValTok{0}\NormalTok{, }\DecValTok{1}\NormalTok{, }\FloatTok{0.7}\NormalTok{), }\AttributeTok{border =} \ConstantTok{NA}\NormalTok{,}
     \AttributeTok{main =} \StringTok{"kNN Imputation for ctrl.1: Overall (Blue) vs Imputed (Red)"}\NormalTok{,}
     \AttributeTok{xlab =} \StringTok{"Intensity"}\NormalTok{)}
\FunctionTok{hist}\NormalTok{(ctrl1\_knn\_imp,}
     \AttributeTok{breaks =}\NormalTok{ brks2,}
     \AttributeTok{col =} \FunctionTok{rgb}\NormalTok{(}\DecValTok{1}\NormalTok{, }\DecValTok{0}\NormalTok{, }\DecValTok{0}\NormalTok{, }\FloatTok{0.7}\NormalTok{), }\AttributeTok{border =} \ConstantTok{NA}\NormalTok{,}
     \AttributeTok{add =} \ConstantTok{TRUE}\NormalTok{)}
\FunctionTok{legend}\NormalTok{(}\StringTok{"topright"}\NormalTok{, }\AttributeTok{inset =} \FloatTok{0.02}\NormalTok{,}
       \AttributeTok{legend =} \FunctionTok{c}\NormalTok{(}\StringTok{"Overall (after kNN)"}\NormalTok{, }\StringTok{"Imputed only (kNN)"}\NormalTok{),}
       \AttributeTok{fill =} \FunctionTok{c}\NormalTok{(}\FunctionTok{rgb}\NormalTok{(}\DecValTok{0}\NormalTok{, }\DecValTok{0}\NormalTok{, }\DecValTok{1}\NormalTok{, }\FloatTok{0.7}\NormalTok{), }\FunctionTok{rgb}\NormalTok{(}\DecValTok{1}\NormalTok{, }\DecValTok{0}\NormalTok{, }\DecValTok{0}\NormalTok{, }\FloatTok{0.7}\NormalTok{)), }\AttributeTok{border =} \ConstantTok{NA}\NormalTok{, }\AttributeTok{cex =} \FloatTok{0.8}\NormalTok{)}
\FunctionTok{grid}\NormalTok{()}
\end{Highlighting}
\end{Shaded}

\begin{center}\includegraphics{assignment01_files/figure-latex/knn_plot_ctrl1-1} \end{center}

\textbf{-\textgreater{} (c): }\textbf{Compare the kNN result to the
distribution-based imputation in terms of mean, SD, number of values
(overall and imputed-only).}

\hypertarget{c-quantitative-comparison}{%
\paragraph{(c) Quantitative
comparison:}\label{c-quantitative-comparison}}

The table contrasts summary statistics for ctrl.1 using the two methods.
Note: kNN uses the filtered dataset (rows where not all three controls
are missing), while the distribution-based method was applied to the
full dataset.

\begin{Shaded}
\begin{Highlighting}[]
\NormalTok{compare\_df }\OtherTok{\textless{}{-}} \FunctionTok{rbind}\NormalTok{(}
  \FunctionTok{data.frame}\NormalTok{(}
    \AttributeTok{method =} \StringTok{"Distribution"}\NormalTok{,}
    \AttributeTok{n =} \FunctionTok{length}\NormalTok{(imp\_ctrl1}\SpecialCharTok{$}\NormalTok{values),}
    \AttributeTok{missing\_before =} \FunctionTok{sum}\NormalTok{(}\FunctionTok{is.na}\NormalTok{(ms\_raw}\SpecialCharTok{$}\NormalTok{ctrl}\FloatTok{.1}\NormalTok{)),}
    \AttributeTok{imputed\_n =} \FunctionTok{length}\NormalTok{(imp\_ctrl1}\SpecialCharTok{$}\NormalTok{imputed),}
    \AttributeTok{mean\_overall =} \FunctionTok{mean}\NormalTok{(imp\_ctrl1}\SpecialCharTok{$}\NormalTok{values),}
    \AttributeTok{sd\_overall =} \FunctionTok{sd}\NormalTok{(imp\_ctrl1}\SpecialCharTok{$}\NormalTok{values),}
    \AttributeTok{mean\_imputed =} \FunctionTok{mean}\NormalTok{(imp\_ctrl1}\SpecialCharTok{$}\NormalTok{imputed),}
    \AttributeTok{sd\_imputed =} \FunctionTok{sd}\NormalTok{(imp\_ctrl1}\SpecialCharTok{$}\NormalTok{imputed)}
\NormalTok{  ),}
  \FunctionTok{data.frame}\NormalTok{(}
    \AttributeTok{method =} \StringTok{"kNN"}\NormalTok{,}
    \AttributeTok{n =} \FunctionTok{length}\NormalTok{(ctrl1\_knn\_all),}
    \AttributeTok{missing\_before =}\NormalTok{ missing\_before\_ctrl1,}
    \AttributeTok{imputed\_n =} \FunctionTok{length}\NormalTok{(ctrl1\_knn\_imp),}
    \AttributeTok{mean\_overall =} \FunctionTok{mean}\NormalTok{(ctrl1\_knn\_all),}
    \AttributeTok{sd\_overall =} \FunctionTok{sd}\NormalTok{(ctrl1\_knn\_all),}
    \AttributeTok{mean\_imputed =} \FunctionTok{mean}\NormalTok{(ctrl1\_knn\_imp),}
    \AttributeTok{sd\_imputed =} \FunctionTok{sd}\NormalTok{(ctrl1\_knn\_imp)}
\NormalTok{  )}
\NormalTok{)}
\NormalTok{compare\_df}
\end{Highlighting}
\end{Shaded}

\begin{verbatim}
##         method    n missing_before imputed_n mean_overall sd_overall
## 1 Distribution 6960            587       587     28.65922   2.925119
## 2          kNN 6733            360       360     28.81819   2.832218
##   mean_imputed sd_imputed
## 1     24.56816   1.312180
## 2     24.96160   1.224823
\end{verbatim}

\textbf{-\textgreater{} (d): }\textbf{What are the advantages and
disadvantages of kNN imputation compared to the distribution-based
approach?}

\hypertarget{d-discussion}{%
\paragraph{(d) Discussion:}\label{d-discussion}}

\begin{itemize}
\item
  \textbf{kNN advantages}: leverages multivariate structure (uses
  similarity across ctrl.1--3), preserves local relationships and can
  adapt to non-Gaussian patterns; may yield more realistic values for
  correlated variables.
\item
  \textbf{kNN disadvantages}: requires tuning \texttt{k}; sensitive to
  scaling and outliers; can bias towards dense regions; depends on
  available neighbors (after filtering) and may be computationally
  heavier on large datasets.
\item
  \textbf{Distribution-based advantages}: simple, fast, controlled
  placement of imputed values in the lower tail to reflect
  left-censoring; easy to explain and reproduce.
\item
  \textbf{Distribution-based disadvantages}: ignores relationships
  between variables; assumes normality and chosen tail/variance; risk of
  under/over-dispersion if parameters are poorly chosen.
\end{itemize}

\end{document}
